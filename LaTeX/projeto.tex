\documentclass[a4paper,11pt,fleqn]{article}

\usepackage[brazil]{babel}
\usepackage[utf8]{inputenc}
\usepackage{xspace}
\usepackage{color}
\usepackage{graphicx}
\usepackage{hyperref}
\usepackage{enumitem}

%\usepackage[shortlabels]{enumitem}


\newcommand{\blue}[1]{\textcolor{blue}{#1}}

\title{Projeto Prático 1\\
Programação Orientada a Objetos - MC302}
\author{Alunos: André Papoti, Bruno Falkenburg, Lucas Ramos, \\
Nicolas França, Sophia Estrêla\\\\
Professora: Esther Colombini\\\\
Unicamp - Instituto de Computação\\}
\date{Abril de 2018}

\begin{document}

\maketitle


\section{Introdução}



Aqui é como Cita~\cite{pfaff2004,pfaff2004full}.

Aqui é como referencia~\ref{s:conceitos} apresentamos os conceitos e definições
relacionados.  Na seção~\ref{s:questoes-hipoteses} enumeramos nossas
hipóteses e na seção~\ref{s:metodos} descrevemos a metodologia que será
seção~\ref{s:atividades-cronograma} apresentamos as atividades e cronograma


\section{Conceitos e Definições}
\label{s:conceitos}


\subsection{Isso é uma subseção}
\label{ss:bst}

: $8
\rightarrow 3 \rightarrow 6 \rightarrow 4$. Dessa forma, com um algoritmo


\begin{figure}[h!]
  \begin{center}
    \includegraphics[scale=0.5]{imagens/bst-example.png}
  \end{center}
  \caption{Exemplo de . Fonte:
    \href{https://en.wg}{Wikipedia}.}
  \label{f:bst}
\end{figure}





\subsubsection{Isso é uma sub-subseção}
\label{p:splay-tree}

execução limitado por $O(\log_2 n)$ para operações de busca, remoção e
suficientemente longa de operações é $O(\log_2 n)$.

% acessa a cada 2 segundos vai ter acesso a conta de forma muito mais rápida que



% Uma árvore B é uma árvore que pode ter $\ell$ elementos em cada nó, e entre
% cada nó temos ponteiros para outros nós com $\ell$ elementos.
\section{Objetivos}
\label{s:questoes-hipoteses}


\section{Métodos}
\label{s:metodos}




\section{Cronograma de Atividades}
\label{s:atividades-cronograma}

\section{Ferramentas Utilizadas}
\label{s:ferramentas}
\subsection{Lagom}
\label{ss:lagom}
O Lagom Framework será utilizado para desenvolver serviços que serão consumidos pelo sistema.
 Entende-se Framework como uma coletânea de códigos genéricos - ou seja, que podem ser utilizados em diversos
  projetos - que têm por finalidade auxiliar o programador no desenvolvimento, servindo de esqueleto
   para o código que será escrito. Ao contrário de bibliotecas, frameworks ditam o fluxo de controle da aplicação.


O framework contém um conjunto de APIs que facilitam ao desenvolvedor o trabalho de escrever microserviços em Java, que
 podem fazer uso de ferramentas já incluídas no Lagom, como servidores para permanência dos dados e compartilhamento de informações
  com outros serviços. Todas as ferramentas utilizadas e os serviços desenvolvidos podem ser inicializados com um único
   comando, ou separadamente.

Microserviço é uma abordagem que visa construir um sistema como um conjunto de pequenos serviços, com funcionalidades
 específicas e completas, e que se comunicam por meios leves, havendo baixa dependência entre os módulos. Este tipo
  de arquitetura colabora com a manutenção da modularidade, permitindo que alterações em serviços específicos não alterem o resultado
   do sistema como um todo. A estrutura dos serviços Lagom segue firmemente este modelo, fortalecido pela separação entre a declaração da interface do
    serviço e a implementação em si.

O Lagom oferece também a possibilidade dos microserviços nele construídos consumirem serviços externos, que não
 precisam seguir a estrutura dos microserviços Lagom. Isto será utilizado, por exemplo, quando for preciso extrair dados de rotas utilizando a API do Google Maps
  para descobrir uma boa rota entre os imóveis que um corretor deseja mostrar a um cliente.

Das APIs que o Lagom fornece, podem-se destacar a Service API, útil para as declarações das interfaces dos serviços desenvolvidos, bem como sua
 implementação, e a Persistence API, que auxilia no controle da persistência de dados. Apesar de não ser o banco de dados padrão, o Lagom
  tem suporte ao PostgreSQL, que será futuramente utilizado.

No momento, há escrito um conjunto de instruções básicas sobre as funcionalidades do Lagom, bem como instruções de instalação e configuração.
 O texto está disponível no arquivo "DocumentacaoLagom.pdf" localizado no diretório raiz do primeiro projeto ("Projeto1/").


\subsection{APIs do Google Maps}
\label{ss:maps}

A Google oferece alguns serviços para utilização do Maps por outros desenvolvedores. Dentre as APIs oferecidas, constam a Directions API,
 cuja finalidade principal é a de encontrar direções entre diferentes localidades; a Distance Matrix API, que calcula tempo e distância entre
  pontos em uma rota, e a Geocoding API, que transforma um endereço em coordenadas e vice-versa.

Para facilitar o uso dessas APIs no código Java, está sendo utilizado o Java Client for Google Maps Services, uma biblioteca desenvolvida pela equipe do
 Google Maps.

Os serviços que a Google oferece serão utilizados para calcular, sob demanda do corretor, uma boa rota entre diversos imóveis que ele deseja mostrar ao
 cliente.

 Até o momento, foi utilizado o cliente Java do Google Maps para calcular uma rota entre dois endereços, e retornar as etapas do movimento em formato
  JSON, para que o resultado possa ser facilmente utilizado em outras aplicações, como numa interface gráfica. Utilizou-se a Directions API, que possibilita
   encontrar mais de uma rota para diversos meios de transporte. Entretanto, no exemplo optou-se por buscar apenas uma rota de carro, configuração que mais se assemelha
    com o futuro uso dos serviços no sistema. Os códigos fonte podem ser encontrados em "Projeto1/Lagom/maps-sem-lagom/maps-testes/src/".

\subsection{PostgreSQL}
\label{ss:postgre}

PostgreSQL é um SGBD (Sistema Gerenciador de Banco de Dados) objeto-relacional. Isso significa que os dados no banco são modelados como entidades relacionadas, semelhantes
 a tabelas, acrescidos de estruturas típicas de orientação a objetos. A linguagem utilizada no PostgreSQL é a SQL.

 Será utilizado para criação e gerenciamento de um banco de dados que armazenará os dados referentes às entidades do sistema.


\subsection{GitHub}
\label{ss:github}

O GitHub é uma plataforma que permite hospedar e compartilhar arquivos, com foco em arquivos de código-fonte.

Por utilizar o Git para controle de versão, os programadores podem trabalhar em ramificações locais do projeto e enviar ao repositório hospedado no GitHub os arquivos
 em que trabalharam, registrando todas alterações e permitindo que outro desenvolvedor que esteja trabalhando no projeto possa permanecer atualizado sobre o progresso
  do outro.

Por conta dos benefícios que esta plataforma traz, o grupo está a utilizando para controle do código - garantindo que todos estejam com versões atualizadas
 e possam com a mesma facilidade revisar e alterar o próprio código ou o de outro membro da equipe - e dos demais arquivos relacionados ao projeto. 

\subsection{Trello}
\label{ss:trello}
Nicolas falará

\subsection{draw.io}
\label{ss:draw}

Para a criação do Diagrama de Classes UML presente neste arquivo foi utilizado o draw.io, uma ferramenta online de criação e edição de diversos tipo de diagramas.
 O site permite integração dos diagramas com o Google Drive, o que pode ser útil para compartilhamento dos diagramas e edição simultânea por mais de um membro.

\bibliographystyle{plain}
\bibliography{refs}


\end{document}
